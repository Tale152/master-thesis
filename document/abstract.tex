\chapter*{Abstract}
\addcontentsline{toc}{chapter}{Abstract}
\begin{spacing}{1.3}
\textit{\small
The idea of Grid Computing originated in the nineties and found its concrete applications in contexts like the SETI@home project where a lot of computers (offered by volunteers) cooperated, performing distributed computations, inside the Grid environment analyzing radio signals trying to find extraterrestrial life.
}\newline

\textit{\small
The Grid was composed of traditional personal computers but, with the emergence of the first mobile devices like Personal Digital Assistants (PDAs), researchers started theorizing the inclusion of mobile devices into Grid Computing; although impressive theoretical work was done, the idea was discarded due to the limitations (mainly technological) of mobile devices available at the time. Decades have passed, and now mobile devices are extremely more performant and numerous than before, leaving a great amount of resources available on mobile devices, such as smartphones and tablets, untapped.
}\newline

\textit{\small
Here we propose a solution for performing distributed computations over a Grid Computing environment that utilizes both desktop and mobile devices, exploiting the resources from day-to-day mobile users that alternatively would end up unused.
}\newline

\textit{\small
The work starts with an introduction on what Grid Computing is, the evolution of mobile devices, the idea of integrating such devices into the Grid and how to convince device owners to participate in the Grid. Then, the tone becomes more technical, starting with an explanation on how Grid Computing actually works, followed by the technical challenges of integrating mobile devices into the Grid. A brief explanation of the MapReduce paradigm, which will be executed over the grid, is then provided. Next, the model, which constitutes the solution offered by this study, is explained, followed by a chapter regarding the realization of a prototype that proves the feasibility of distributed computations over a Grid composed by both mobile and desktop devices. To conclude future developments and ideas to improve this project are presented.
}
\end{spacing}
