\section{Architecture}
This section describes the software architecture proposed by this project in order to realize the Transparent scheduling model over heterogeneous devices and, consequently, satisfy the requirements and use cases previously described.

\textit{Figure \ref{fig:architecture_complete}} provides a complete view of the Architecture; being a complex project based on the interaction between multiple distributed entities, the schema will be divided in three areas (easier to understand) that will be discussed in the following sections:
\begin{itemize}
    \item Cloud Services area (\textit{section \ref{cloud_services_area}})
    \item Contributor area (\textit{section \ref{contributor_area}})
    \item Customer area (\textit{section \ref{customer_area}})
\end{itemize}
\begin{figure}[!ht]
    \centering
    \includegraphics[width=\linewidth]{document/chapters/chapter_6/images/architecture_complete.jpg}
    \caption{Complete view of the Architecture}
    \label{fig:architecture_complete}
\end{figure}

\subsection{Cloud Services Area}\label{cloud_services_area}
The Cloud services follow a Hexagonal architecture composed by a multitude of microservices (\textit{Figure \ref{fig:architecture_cloud_services}}). Each Microservice is accessed through communication interfaces (whether Rest APIs or Web Sockets, depending on the particular microservice needs) and belong to one of the following layers:
\begin{itemize}
    \item \textbf{Business logic}\\
    Microservices that expose core business logic for Grid functionalities realization and data managing. Entities that  reside here are protected, meaning that they are isolated from the outside and their functionalities can only be accessed through the entities placed in the Adapters layer.
    \item \textbf{Adapters}\\
    Microservices that expose functionalities accessed by the entities residing in the Contributor and Customer area; such functionalities are realized through combining the services offered by entities residing in the Business logic layer.
\end{itemize}
\begin{figure}[!ht]
    \centering
    \includegraphics[scale=0.35]{document/chapters/chapter_6/images/architecture_cloud_services.jpg}
    \caption{Architecture: Cloud Services}
    \label{fig:architecture_cloud_services}
\end{figure}

\subsubsection{Business Logic}
\begin{itemize}
    \item \textbf{Grid Master Service}\\
    The main coordinator in the Grid system. It dynamically creates/removes Broker Service instances in order to sustain and balance the traffic generated by Nodes and Invoking Endpoints connected to the current instances of Broker Service; such instances needs to be contacted by Nodes but, being dynamically instantiated, they do not possess a static address. As a consequence of that, the Grid Master Service (that knows such addresses, being the creator of the instances) is connected to the Broker Discovery System (which will provide a Broker Service instance address to a connecting Node).

    \textit{Figure \ref{fig:master_grid_load_balancing}} shows a high level view of the connection between Grid Master, Brokers and Nodes while also providing an example of load balancing. When the new Node [N11] wants to connect to the Grid in order to Contribute, given that no Broker instance is able to handle the Node connection, the Grid Master instantiates [B2] to which [N11] will connect and part of the load handled by [B1] (just [N9] in this example) will be redirected to.
    \begin{figure}[!ht]
        \centering
        \includegraphics[width=\linewidth]{document/chapters/chapter_6/images/master_grid_load_balancing.jpg}
        \caption{Grid Master Service load balancing}
        \label{fig:master_grid_load_balancing}
    \end{figure}

    Lastly, the Grid Master Service is also connected to the Grid Services Gateway Service; through this last Cloud Service, the Invoking Endpoints request the execution of Grid Services. Thus, the Grid Master (collaborating with the Broker Service instances) will provide the Resources needed to execute the requested Grid Service.

    Concluding, this Cloud Service's importance is vital to the functioning and scalability of the Grid, requiring to expose communication interfaces for the discovery of Brokers, Resources obtainment and Grid coordination.

    \item \textbf{Customer Data Service}\\
    REST server that provides APIs used to read and write data used to uniquely identify a Customer inside the system. It does not contain data about payments or logs about Grid events that involve the Customer since those are handled by the Accounting Service and the Log Service respectively.\\
    Customers are, numerically speaking, considerably less compared to Contributors; this results in less frequent invocations of this Cloud Service's APIs and a far smaller volume of data to persist. Regarding the CAP theorem, it is then important to focus on a database technology that can grant Consistency and Partition Tolerance sacrificing Availability (e.g. MongoDB, BigTable, etc...).

    \item \textbf{Contributor Data Service}\\
    REST server that exposes APIs used to read and write data used to uniquely identify a Contributor and its devices inside the system. The same rules used in the Customer Data Service, regarding the handling of logs and payments data, also hold here; when it comes to the CAP theorem application, on the contrary, this Cloud Service requires AP database technologies (e.g. DynamoDB, Cassandra, etc...).

    \item \textbf{Accounting Service}\\
    REST server exposing APIs to perform and record the history of monetary transactions, involving both Customers and Contributors (i.e. Fees payments and Rewards Redemptions) inside the Grid; Consistency and Partition Tolerance here are key requirements.

    \item \textbf{Log Service}\\
    REST server providing APIs used to read and write unmodifiable logs about Contributions and Grid Services Invocations; this Cloud Service is particularly important to both monitor what is happening inside the Grid and also correctly identify which Node has performed a Contribution and how much of said Contribution it has done. Access speed is the most relevant factor here, making it acceptable to have eventual consistency but not delays; then, AP database technologies are required here. In particular, said AP database technology should use an RDF model (also known as Triplestore) which is particularly suited for log data.
\end{itemize}

\subsubsection{Adapters}
\begin{itemize}
    \item \textbf{Broker Service}\\
    Cloud Service that acts as middleware between the Grid Master and the Nodes (\textit{figure \ref{fig:master_grid_load_balancing}}). There are as many instances as needed to sustain the load of the connections to the Nodes; instances are created and removed by the Grid Master taking in account the geographical location of said Nodes in order to reduce latency for the Broker-Node connection.
    The Broker executes the coordination commands given by the Grid Master while also managing the Nodes connected through its communication channels exposed by its Web Socket.
    
    Let us take, for example, a Grid Service Invocation: the Grid Master contacts a Broker Service instance that is geographically convenient in relationship to the location of the Invoking Endpoint; the Grid Master will delegate to the selected Broker Service instance the responsibility of gathering adequate Resources for the execution of the particular Grid Service requested. The Broker will then spread the request of said Resources to its connected Nodes and gather the responses of the ones that are adequate and available. The Broker will group the info needed to contact such nodes and forward it to the Grid Master that will then be responsible to forward in turn to the requestor.

    \item \textbf{Broker Discovery Service}\\
    This REST server has just one simple responsibility: it provides to a Node the address of a Broker Service in order to make possible a connection between them. This discovery mechanism is necessary since, as already stated, the Broker Service instances are dynamically created and thus they do not possess a static address; on the contrary the Broker Discovery Service will necessarily need a static address known by the Nodes. The trafic directed to this Cloud Service and its computational load tend to be minimal and requires only one instance but, as time goes on and the contributing user base increases, new static instances can be added easily.

    \item \textbf{Grid Services Gateway Service}\\
    In order for Invoking Endpoints to access Resources, they need interact with this REST server; it exposes APIs for a standardized and parameterized access to Resources through the Grid Service abstraction, meaning that a Customer expresses its request in terms of what Grid Service it wants to execute, not in terms of single Resources.
    
    This Cloud Service is connected via Web Socket to the Grid Master Service in order to gather the necessary Resources but, before that, the Grid Services Gateway Service needs to contact the Accounting Service in order to execute the Fee payment. Lastly, the Cloud Service is also connected to the Log Service in order to register the Grid Service invocation and the consequent usage of Resources happened during the computation.

    Similarly to the Broker Discovery Service, the number of static instances easily can vary in the project's lifecycle. 

    \item \textbf{Authentication Service}\\
    TODO
    \item \textbf{Billing Service}\\
    TODO
    \item \textbf{Historian Service}\\
    TODO
\end{itemize}
\subsection{Contributor area}\label{contributor_area}
TODO

\begin{figure}[!ht]
    \centering
    \includegraphics[width=\linewidth]{document/chapters/chapter_6/images/architecture_contributor.jpg}
    \caption{Architecture: Contributor-relevant view}
    \label{fig:architecture_contributor}
\end{figure}

\subsubsection{Node}
TODO
% High level grid draw.io: High level grid and Load Balancing

\subsubsection{Contributing Endpoint}
TODO
% Focus on interfaces and device-specific stuff

\subsubsection{Contributor Dashboard}
TODO
% Mockups

\subsection{Customer area}\label{customer_area}
TODO
\begin{figure}[!ht]
    \centering
    \includegraphics[width=\linewidth]{document/chapters/chapter_6/images/architecture_customer.jpg}
    \caption{Architecture: Customer-relevant view}
    \label{fig:architecture_customer}
\end{figure}

\subsubsection{Invoking Endpoint}
TODO
% High level grid draw.io: something similar to MapReduce but for a general task

\subsubsection{Customer Custom Application}
TODO
% focus on sdk

\subsubsection{Customer Dashboard}
TODO
% Mockups