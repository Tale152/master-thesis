\section{MapReduce Service}\label{mapreduce_service}
This section will explore the \textbf{logical design of the MapReduce operation performed as a Grid Service}. The process is modeled using three \textbf{Petri Nets}, each one \textbf{providing the point of view of one among the possible entities types involved}: Map Worker, Reduce Worker and MapReduce Master.

Every Petri Net that will be presented \textbf{assumes that the Resources' retrieval process} (viewed in \textit{figure \ref{fig:use_cases_satisfaction_node_contribution}}) \textbf{has already been performed, allowing to focus on the actual MapReduce process itself}.
Furthermore, in every Petri Net snapshot provided, transitions that can be transitioned (in that particular configuration of the places) are highlighted in red to facilitate its understanding.
  
\subsection{Map Worker}
The process starts with the Map Worker establishing a connection with the MapReduce Master (\textit{figure \ref{fig:map_worker_petri_net_1}}).

\vspace{5mm}

\begin{figure}[!ht]
    \centering
    \includegraphics[width=\linewidth]{document/chapters/chapter_6/images/map_worker_petri_net_1.png}
    \caption{Map Worker - Start}
    \label{fig:map_worker_petri_net_1}
\end{figure}

Ignoring, for the moment, the Disconnection' transition, the Map Worker receives the Map function from the MapReduce Master, obtaining a token for the Execution place and one for the Master Connected'' place. The distinction between the concepts of connection and execution is useful here since it is not always necessary to be connected to the Grid Master in order to execute some operations; through this distinction, Resources can be used efficiently even if a temporary disconnection occurs.

\textit{Figure \ref{fig:map_worker_petri_net_2}} shows how, once the map function is received, the MapReduce Master starts to send the location where the data splits can be retrieved; the Map Worker retrieves the specified data and applies the map function to it producing the intermediate results (IR) as output.

\vspace{5mm}

\begin{figure}[!ht]
    \centering
    \includegraphics[width=\linewidth]{document/chapters/chapter_6/images/map_worker_petri_net_2.png}
    \caption{Map Worker - Mapping}
    \label{fig:map_worker_petri_net_2}
\end{figure}

As can be seen from the operations executed, the Map Worker acts following the instructions received by the MapReduce master. While mapping, a data split can be assigned to multiple Map Workers, in order to increase fault tolerance; as a direct consequence, it is possible that a received split location is mapped by another Map Worker. Once the IR is already obtained on a particular split, the Map Worker sends a split revoked' instruction to every Map Worker assigned to that split. The two different split revoked transitions possess a conceptual difference:
\begin{itemize}
    \item Split revoked'\\
    Here, the Map Worker that receives this instruction cannot possibly have applied the map function to the split, since the split revoked instruction is received only after the IR (produced by the mapping of that particular split) is reduced by a Reduce Worker. Practically speaking, the Map Worker drops the split, removing it from memory or avoiding downloading it entirely.
    \item Split revoked''\\
    In this situation, the Map Worker already performed the map operation on the considered split. Whether the IR that will be used for the reduce operation is actually the one locally produced or not is irrelevant; the direct consequence of receiving a split revoked instruction here is to remove from memory the IR. This is because the MapReduce Master can ask multiple times to the Map Worker to send the IR to a Reduce Worker (\textit{figure \ref{fig:map_worker_petri_net_3}}), requiring to maintain in memory (or on disk) the IR until the split revoked instruction is received.
\end{itemize}

\vspace{2mm}

\begin{figure}[!ht]
    \centering
    \includegraphics[width=\linewidth]{document/chapters/chapter_6/images/map_worker_petri_net_3.png}
    \caption{Map Worker - Intermediate Result sent}
    \label{fig:map_worker_petri_net_3}
\end{figure}

\vspace{2mm}

At any given time, the Map Worker can lose the connection with the MapReduce Master (Disconnection' and Disconnection'' transitions); before receiving the Map function, the process is necessarily locked until the connection is established once again or a timeout is reached, letting to a local failure. On the contrary, if the token on the execution place is present, on a disconnected state the Map Worker is still able to apply the map function on a split that is locally available; on the other hand, a timeout leading to a failure would also remove the execution token, effectively stopping the entire process.
\vspace{8mm}

When the MapReduce operation is completed, the Master will send a completion instruction that will result in the removal of the connection and execution tokens, correctly concluding the execution (\textit{figure \ref{fig:map_worker_petri_net_4}}).

\begin{figure}[!ht]
    \centering
    \includegraphics[scale=0.45]{document/chapters/chapter_6/images/map_worker_petri_net_4.png}
    \caption{Map Worker - Completion}
    \label{fig:map_worker_petri_net_4}
\end{figure}

\subsection{Reduce Worker}
The first steps of the reduce process are the same as the map ones: a connection to the MapReduce Master is established and a reduce function is received, gaining a connection token and an execution token (used for the same Resources' utilization optimization principle); here a disconnection can also occur and, if a timeout is reached, the process ends with a failure.

The MapReduce Master sends the data about which IRs constitute a region (\textit{figure \ref{fig:reduce_worker_petri_net_1}} shows an example of a region composed by 10 IRs); the Reduce Worker also receives, progressively, the IRs from the Map Workers that have completed the execution of a map function and are instructed to send said data to a particular Reduce Worker. Once the all the IRs belonging to a region are received, the Reduce Worker performs the reduce function on said data, producing a result that is momentarily stored locally.

\begin{figure}[!ht]
    \centering
    \includegraphics[scale=0.44]{document/chapters/chapter_6/images/reduce_worker_petri_net_1.png}
    \caption{Reduce Worker - Local result}
    \label{fig:reduce_worker_petri_net_1}
\end{figure}

A result correctly sent to the Master is deleted locally. A Reduce Worker handles multiple regions until the MapReduce process is completed; once the Master sends to the Worker a completion instruction the execution ends correctly (\textit{figure \ref{fig:reduce_worker_petri_net_2}}).

\begin{figure}[!ht]
    \centering
    \includegraphics[scale=0.44]{document/chapters/chapter_6/images/reduce_worker_petri_net_2.png}
    \caption{Reduce Worker - Completion}
    \label{fig:reduce_worker_petri_net_2}
\end{figure}

\subsection{MapReduce Master}
The Petri Net describing the process performed by the MapReduce Master focuses only on a single region, meaning that this operation needs to be performed (concurrently) for every data region. 
\begin{figure}[!ht]
    \centering
    \includegraphics[width=\linewidth]{document/chapters/chapter_6/images/master_petri_net_1.png}
    \caption{MapReduce Master - Recruitment}
    \label{fig:master_petri_net_1}
\end{figure}

For this particular example, the region is constituted by 4 splits and the Customer required a quantity of Resources that resulted in the Contribution of 5 Map Workers; a region is handled by a single Reduce Worker, so there is only one token representing it.

Initially, the Map and Reduce Workers are required but not connected (Requested MWs and Requested RW places); they can transition to a connected state but, if something goes wrong, they can also disconnect. \textit{Figure \ref{fig:master_petri_net_1}} shows a situation where 3 Map Workers and the Reduce Worker have joined. Workers can continue to join and leave until the MapReduce computation for the region is completed, removing the execution token.

Having at least one MW available, the mapping process can begin (\textit{figure \ref{fig:master_petri_net_2}}). A split belonging to the region is assigned to an available MW which will perform the Map operation on it (it is assumed that the Worker possesses the map function, abstracted for simplicity). If the mapping happens successfully, the region's IRs are accumulated; in case that a Map Worker is lost during the map execution, the split needs to be computed by another worker and is placed back in the regions splits place (and the another token in the requested MWs is also placed).
It can also happen that a region IR is lost due to the disconnection of the Map Worker that is still keeping the result in memory; this occurrence also adds again said split in the region splits that still needs to be mapped (since its IR is unreachable).

\begin{figure}[!ht]
    \centering
    \includegraphics[width=\linewidth]{document/chapters/chapter_6/images/master_petri_net_2.png}
    \caption{MapReduce Master - Mapping process}
    \label{fig:master_petri_net_2}
\end{figure}

Once all the region's split are mapped to intermediate results (\textit{figure \ref{fig:master_petri_net_3}}), the reduce process can start (assuming that a RW is connected).

\begin{figure}[!ht]
    \centering
    \includegraphics[width=\linewidth]{document/chapters/chapter_6/images/master_petri_net_3.png}
    \caption{MapReduce Master - Region Mapping completed}
    \label{fig:master_petri_net_3}
\end{figure}

While the reduce function is being applied, the region token is removed due to the fact that no more splits for this region needs to be mapped (\textit{figure \ref{fig:master_petri_net_4}}). A failure can also happen here, since the Reduce Worker can disconnect during the reduce function execution; in this case, the region token is restored and a new Reduce Worker is required.

\vspace{5mm}

\begin{figure}[!ht]
    \centering
    \includegraphics[width=\linewidth]{document/chapters/chapter_6/images/master_petri_net_4.png}
    \caption{MapReduce Master - Reducing process}
    \label{fig:master_petri_net_4}
\end{figure}

If the reduce function is completed successfully (as shown in \textit{figure \ref{fig:master_petri_net_5}}), the result for the considered region is obtained. As stated before, the overall MapReduce execution will end once every region is computed, providing to the Customer all the regional results. 

\begin{figure}[!ht]
    \centering
    \includegraphics[width=\linewidth]{document/chapters/chapter_6/images/master_petri_net_5.png}
    \caption{MapReduce Master - Region completed}
    \label{fig:master_petri_net_5}
\end{figure}