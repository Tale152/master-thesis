\section{Challenges derived from including mobile devices}\label{challenges_derived_from_mobile_devides}
While the intrinsic challenges described in \textit{section \ref{intrinsic_challenges_of_grid_computing}} remain still valid, the inclusion of mobile devices in the Grid arises new issues that have to be taken in consideration while designing the system. Some challenges have been resolved by the evolution of technology and the new behaviors adopted by mobile users (\textit{section \ref{previous_works_and_limitations}}) but other ones still need to be dealt with.

\subsection{Network instability}
TODO

\subsection{Battery consumption}
Having devices that run on battery and are not connected to a reliable power source can be a source of problems for the Grid. \textbf{Sudden disconnections or failures has to be monitored anyway} since devices connected to a power source may be subjected to a sudden interruption of energy from the energy source or, more in general, can incur in any kind of error. \textbf{The problem with mobile devices is more about the increased possibility of unplanned interruptions due to energy exhaustion of the battery}.
There are \textbf{five main factors} that \textbf{affect power usage} in mobile devices \cite{mobile_power_consumption}:
\begin{enumerate}
    \item \textbf{Hardware} (CPU, GPU, memory, storage, display, sensors);
    \item \textbf{Signaling and networking modules} (cellular network, Bluetooth, hotspot, Wi-Fi, GPS, FM radio);
    \item \textbf{Software} (operating system, background applications)
    \item \textbf{Usage patterns} (calling, internet browsing, social networks usage, gaming, music playback, video playback, running heavy applications);
    \item \textbf{Other factors} (inter-device communication, heating, aging and faulty battery).
\end{enumerate}
\textbf{How long the battery of a mobile device depends on a combination of such factors}, ranging from days to (in extreme cases) only hours.

There are a set of \textbf{possible countermeasures} that limit the participation to the Grid only to devices that respect certain prerequisites that aims to mitigate this problem:
\begin{itemize}
    \item \textbf{External power source}\\
    The first obvious countermeasure is to only allow devices to contribute to the Grid while connected to a reliable power source. While certainly effective and totally feasible (since commonly mobile devices tend to be charged every day), on the other hand this can limit the percentage of contributors active in a certain moment;
    \item \textbf{Battery health status}\\
    Another method is checking for the device's battery health, allowing contribution to the Grid while disconnected from an external power source only to devices that have healthy battery units.
    \item \textbf{Battery percentage}\\
    The final countermeasure is allowing battery-powered contribution only when the device's battery is above a certain battery percentage.
\end{itemize}
The \textbf{best solution} to mitigate this problem is \textbf{a combination of the three countermeasures}, allowing healthy devices to contribute while above a certain battery percentage and limiting devices with an aged battery to only contribute while connected to a power source.

On a final note, what discussed in this section can be mostly applied also to laptops, even though people usually tend to use them while connected to a power source.

\subsection{Limited resources}
TODO

\subsection{Even greater focus on security}
\textbf{As the number of smartphone increases}, the number of computers used by people are gradually declining (\textit{figure \ref{fig:global_sales_of_pcs_and_smartphones}}); as a direct consequence of this phenomenon, \textbf{malicious attack efforts are now being redirected to exploiting mobile devices' vulnerabilities}.
\begin{figure}[H]
    \centering
    \includegraphics[width=\linewidth]{document/chapters/chapter_3/images/mobile_vulnerabilities.jpg}
    \caption{Number (left) and distribution (right) of known vulnerabilities on iOS and Android from 2007 to 2019 \cite{mobile_security}}
    \label{fig:mobile_vulnerabilities}
\end{figure}
\textit{Figure \ref{fig:mobile_vulnerabilities}} shows how \textbf{Android has more vulnerabilities compared to iOS}. This can be explained for the following two reasons:
\begin{itemize}
    \item \textbf{Android devices are more diffused}\\
    Attackers tend to attack these devices more since the pool of individual units is larger (as shown in \textit{image \ref{fig:os_market_share_2012_to_2021}}) and thus obtaining a profit is more likely.
    \item \textbf{iOS has more strict security measures} \cite{mobile_security}\\
    One of the byproducts of designing Android to be able to run a vast number of different hardware is the fact that compromises have to be made in order to grant compatibility. On the other hand, iOS runs on only a limited number of models, making it easier to design more effective security mechanisms. 
\end{itemize}
Through the \textbf{Linux Kernel}, Android resources are managed and protected. Furthermore, Android employs an \textbf{Application Sandbox} security mechanism, meaning that each application is run with a unique ID and the \textbf{processes are executed in spaces separated from each other and from the Kernel}, therefore preventing an application to interfere with the functioning of another. Despite this, Android employs an \textbf{authorization-based mechanism for accessing resources}; while \textbf{a process that has not the correct authorizations cannot access determinate resources}, the greatest weakness of this mechanism is the fact that \textbf{authorizations are granted by the user} which, usually, is not really aware of the implications of such choices. Application, consequently, can launch \textbf{various types of attacks depending on the type of authorizations} that they obtained. Another weakness of Android systems is the fact that \textbf{Application provenance is not guaranteed}; the Play Store's digital certificate can be easily obtained from malicious parties since Google only requires a fee paid by a credit card to get it (and payments might be done with a stolen credit card). Applications can also be installed from third parties, making it easier to introduce malicious software that is completely unchecked.

On the other hand, \textbf{iOS} utilizes a similar Sandbox system, but \textbf{authorizations are completely managed by the OS}, removing the possibility of human errors. From the Application provenance perspective, iOS' applications are only distributed by the App Store; developers have to register, pass a \textbf{vetting process} that evaluates the Application which, in case no harm is detected, is \textbf{digitally signed} and only then published on the store. This process makes more difficult the introduction of harmful applications.
\begin{figure}[H]
    \centering
    \includegraphics[scale=1.2]{document/chapters/chapter_3/images/mobile_vulnerabilities_severity.jpg}
    \caption{iOS and Android vulnerabilities severity score from 2015 to 2019 \cite{mobile_security}}
    \label{fig:mobile_vulnerabilities_severity}
\end{figure}
Even though iOS has the advantage when it comes to sheer number of security breaches, \textbf{with time the severity of Android's breached has decreased, becoming less severe than the iOS' ones} (\textit{figure \ref{fig:mobile_vulnerabilities_severity}}).

This data highlights how it is particularly important to focus on the security of the Grid's Fabric Layer (\textit{section \ref{fig:fabric_layer}}) while dealing with mobile devices because, in the current panorama, they are the greatest source of security vulnerabilities, especially considering that the information inside them has great value for attackers.

\subsection{Compatibility issues}
TODO
