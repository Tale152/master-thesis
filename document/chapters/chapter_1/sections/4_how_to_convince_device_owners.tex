\section{How to convince device owners}
Since this work assumes the voluntary contribution to the Grid by users, said people have to be convinced to participate in the project; alternatively, as much as the project allows to reach a great level of scalability, without any physical machines, no tasks can be performed, making it useless. This section discusses ways and prerequisites to incentivize users to participate in the dynamically allocated Grid.

\subsection{Zero-effort configuration}
In order to \textbf{reduce as much as possible the barrier of the active effort} needed to participate in the Grid, users should be presented with a configuration as easy as possible, especially considering that the vast majority of users do not want to be bothered with technical details. The goal is possibly to just \textbf{require an initial guided installation} (using easy to access methods such as applications stores and executables) \textbf{and then never require direct input from the contributor}.

While the \textbf{possibility of partially customizing the setup} must be offered, it should be \textbf{as minimal and easy as possible}, offering possibilities like enabling/disabling participation while using mobile data, while the device is alimented using only the battery and possibly a scheduling, depending on the time of the day. 

\subsection{Unnoticeable impact: smart resource management}
\textbf{It is important that}, while contributing to the Grid, \textbf{the end user does not experience slowdowns} resulting in a worsening of the user experience \textbf{during the normal activities} that the user performs; in order to achieve this, the application running on the device must implement a \textbf{smart resource management system} that adjusts resource usage depending on the current utilization of them by the user. If this mechanism is not well implemented, there could be a risk of the user stopping its contribution because of the discomfort created.

\subsection{Security and privacy}
One of the most important factors is \textbf{granting security and privacy inside the Grid}. This is important for multiple reasons:
\begin{itemize}
    \item \textbf{The end user wants to feel safe }participating in the project, being sure that their devices and personal data do not get compromised;
    \item \textbf{The entities that request the services offered by the Grid also want to feel safe} that their sensitive data (including the results of the computations performed inside the Grid) will not be accessible to non-authorized third parties;
    \item \textbf{The stakeholders of the project do not want to be involved in legal actions} deriving from a breach in security.
\end{itemize}

This requires a particular effort on security in every component of the systems and in the communications that are performed over the network.

\subsection{Fair share business model}\label{fair_share_business_model}
\textbf{Contributors} to the Grid must be \textbf{economically rewarded in order to be incentivized to offer their devices} to the project. Hence, a \textbf{fair share business model} will be useful to this purpose: \textbf{users receive a fraction of the earnings} obtained by the payments of the requestors of Grid services; \textbf{the reward is proportional to the contribution made by the device} that participated to offer the requested service.

This business model allows the owners of the Grid to make a profit (while needing only to sustain the costs of the machines used to connect the machines of the requestors and the nodes of the Grid) while also rewarding contributors of the Grid for the usage of their computational resources (while not needing an active effort other than the initial setup).
