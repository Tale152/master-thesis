\section{How to convince device owners}
Since this works assumes the voluntary contribution to the Grid by users, said people have to be convinced to participate in the project; alternatively, as much as the project allows to reach a great level of scalability, without any physical machines no tasks can be performed making it useless. This section discusses ways and prerequisites to incentivize users to participate the dynamically allocated Grid.

\subsection{Zero-effort configuration}
To reduce as much as possible the barrier of the active effort needed to participate in the Grid, users should be presented with a configuration as easy as possible, especially considering that the vast majority of users do not want to be bothered with technical details. The goal is possibly to just require an initial guided installation (using easy to access methods like applications stores and executables) and then never require direct input from the contributor.

While the possibility of partially customizing the setup must be offered, it should be as minimal and easy as possible, offering possibilities like enabling/disabling participation while using mobile data, while the device is alimented using only the battery and possibly a scheduling depending on the time of the day. 

\subsection{Unnoticeable impact: smart resource management}
It is important that, while contributing to the Grid, the end user does not experience slowdowns resulting in a worsening of the user experience during the normal activities that the user performs; in order to achieve this, the application running on the device must implement a smart resource management system that adjusts resource usage depending on the current utilization of them by the user. If this mechanism is not well implemented, there could be a risk of the user stopping its contribution because of the discomfort created.

\subsection{Security and privacy}
TODO

\subsection{Fair share business model}
TODO
