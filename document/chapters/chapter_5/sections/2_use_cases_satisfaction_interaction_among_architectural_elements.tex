\section{Use cases satisfaction: entities interactions}\label{use_cases_satisfaction}
This section will show how the entities introduced in the Architecture just presented actually cooperate with each other in order to satisfy the use cases previously discussed in \textit{section \ref{use_cases}}. 

\subsection{Contributor}
\begin{itemize}
    \item \textbf{Register}\\
    \begin{figure}[!ht]
        \centering
        \includegraphics[scale=1]{document/chapters/chapter_5/images/use_cases_satisfaction_contributor_register.jpg}
        \caption{Contributor registration}
        \label{fig:use_cases_satisfaction_contributor_register}
    \end{figure}

    A Contributor registers to the Grid system creating an account using the Contributor Dashboard; once it has inputted the necessary data, the Dashboard will contact the Authentication with the dedicated API that, in turn, will contact the Contributors Data Service, actually completing the registration process.

    \item \textbf{Login}\\
    When a Contributor performs a Login, both in the Dashboard case and in the Contributing Endpoint, it gets a token that will be used for every subsequent operation for both authenticating the communication with other entities and, at the same time, providing some useful data that will be used by said entities.
    
    Since the token is a prerequisite for every further interaction it will be omitted for the sake of simplicity in the subsequent single use case satisfaction discussions. 
     
    \begin{itemize}
        \item \textbf{\textit{Login to a Dashboard}}\\
        \begin{figure}[!ht]
            \centering
            \includegraphics[scale=1]{document/chapters/chapter_5/images/use_cases_satisfaction_contributor_dashboard_login.jpg}
            \caption{Contributor Dashboard login}
            \label{fig:use_cases_satisfaction_contributor_dashboard_login}
        \end{figure}

        The login process performed by a Dashboard is very similar to the registration one, diverging only in the invocation of a different specific API.
        \vspace{4mm}

        \item \textbf{\textit{Login from any Contributing Endpoint able to Contribute}}\\
        \begin{figure}[!ht]
            \centering
            \includegraphics[scale=0.8]{document/chapters/chapter_5/images/use_cases_satisfaction_node_login.jpg}
            \caption{Node login}
            \label{fig:use_cases_satisfaction_node_login}
        \end{figure}

        Here, the Contributor Application is the one actually triggering the Login in its integrated Node; then, the Node's login differs from the Dashboard's one calling another dedicated API that also requires to provide some additional info that will uniquely identify the device.
        
        The Customer only needs to make the active effort of registering its account; every Contributing Endpoint will be automatically added to its account if the credentials are correct and no matching device for its account is found.
        One important thing to notice is that the token does not exit the Node's scope.

    \end{itemize}

    \item \textbf{Passively Contribute offering a compatible device}\\
    When it comes to Contribution, the first step that actually needs to be performed is for a Node to connect to the Grid.
    \vspace{12mm}
    \begin{figure}[!ht]
        \centering
        \includegraphics[width=\linewidth]{document/chapters/chapter_5/images/use_cases_satisfaction_node_grid_connection.jpg}
        \caption{Connection to the Grid}
        \label{fig:use_cases_satisfaction_node_grid_connection}
    \end{figure}
    \vspace{5mm}

    First, independently of the Node, the Grid Master Service creates an instance of a Broker Service (there is always at least one instance of such Cloud Service) and, upon creating a connection with it, saves its address; said address is also sent to the Broker Discovery Service that, consequently, will always locally know the addresses of the Broker Service's instances.

    The actual connection to the Grid is triggered automatically by the Contributing Endpoint once the prerequisites are met (an available Internet connection is present, the device's battery is above a certain threshold, etc...). The Node will proceed to contact the Broker Discovery Service that (without needing to contact the Grid Master Service for every new Node request) then provides the address of the more geographically convenient Broker Service instance; with the information obtained, the Node completes the connection process contacting [B1]. 

    Now that the Node is connected to the Grid, the actual Contribution can happen. In order to have a clear understanding of the full picture, a Grid Service Invocation needs to be explained alongside the Node's Contribution.
    \vspace{10mm}

    \begin{figure}[!ht]
        \centering
        \includegraphics[width=\linewidth]{document/chapters/chapter_5/images/use_cases_satisfaction_node_contribution.jpg}
        \caption{Node Contribution}
        \label{fig:use_cases_satisfaction_node_contribution}
    \end{figure}

    The process starts with the Invoking Endpoint requesting a Grid Service to the Grid Services Gateway Service that, will first retrieve the Customer's payment method data (omitted in \textit{figure \ref{fig:use_cases_satisfaction_node_contribution}} for simplicity) and then contact the Accounting Service to pay the Fee required for the Grid Service Invocation; then, the Grid Services Gateway Service will contact the Grid Master Service in order to obtain the address of a Broker Service instance.

    The Invoking Endpoint, now having B1's address, contacts it and establishes a connection that, once performed, allows the Invoking Endpoint to request Resources. B1 broadcasts the request to all its Nodes, specifying the requirements needed for the Grid Service requested. The Nodes that are currently available, possess compatible Access Policies and do satisfy the requirements, take charge of the request and contact Invoking Endpoint and, if the Endpoint accepts it, a connection is created. Finally, the Invoking Endpoint is able to send Tasks that the Node will perform.

    The process describes an Invoking Endpoint to Node connection; when the connection needs to be established between Nodes, the process is very similar. First the Invoking Endpoint to Node connection is created; then, another Resource request will ask for a Contribution to the Nodes but the exchanged messages (containing the Master Node address) will specify that a Node to Node connection is required. Depending on the particular Grid Service then, the Master Node itself will provide the Tasks to its Slave Nodes.

    \begin{itemize}
        \item \textbf{\textit{Accumulate Rewards}}\\
        \begin{figure}[!ht]
            \centering
            \includegraphics[scale=0.6]{document/chapters/chapter_5/images/use_cases_satisfaction_rewards_accumulation.jpg}
            \caption{Contributor's Rewards accumulation}
            \label{fig:use_cases_satisfaction_rewards_accumulation}
        \end{figure}

        Upon a Node's Contribution, the Invoking Endpoint groups Contribution info and sends them to the Historian Service that will then proceed to store them through the Log Service.

        \item \textbf{\textit{Manually start/stop Contribution from the Contributing Endpoint}}\\
        A Contributor can manually stop its Contributing Endpoint from performing further Contributions; this results in that particular Contributing Endpoint interrupting the connection (established in \textit{figure \ref{fig:use_cases_satisfaction_node_grid_connection}}) with a Broker Service instance.
        This action is performed by the Contributor's input in the running Contributing Endpoint that will simply invoke a Node's function instructing it to stop the connection and, consequently, never connecting again to the Grid until the Contributor allows it again.

        \item \textbf{\textit{Configure Access Policies to the Contributing Endpoint's Resources}}\\
        When a Node receives a broadcasted Contribution request by the Broker Instance which is connected to, in order to decide whether to take charge of the request or not, a series of conditions are evaluated; one of such conditions is the compatibility with the local Access Policies configured for that particular Contributing Endpoint. 
        A Contributor, interacting with the Contributing Endpoint, can manually change the Access Policies that, from that moment, will be checked by the Node in its decision process. 

        \item \textbf{\textit{Check if Contributing Endpoints are currently Contributing}}\\
        The Node exposes functions that tell the current status regarding the Grid connection and if a Contribution to a Task is currently being performed; such information is displayed in the Contributing Endpoint's GUI.

    \end{itemize}
    \item \textbf{Manage Contribution data through a Dashboard}
    \begin{itemize}
        \item \textbf{\textit{Check past Contributions}}\\
        \begin{figure}[!ht]
            \centering
            \includegraphics[scale=0.65]{document/chapters/chapter_5/images/use_cases_satisfaction_past_contributions_check.jpg}
            \caption{Contributor's past Contributions check}
            \label{fig:use_cases_satisfaction_past_contributions_check}
        \end{figure}

        The Contributor Dashboard contacts the Historian Service, asking for the past contributions linked to the Contributor's Contributing Endpoints; the Historian Service contacts the Log service where the data is stored (see \textit{figure \ref{fig:use_cases_satisfaction_rewards_accumulation}}). Finally, the information is retrieved and then shown in the Contributor Dashboard.

        \item \textbf{\textit{Check Rewards Balance}}\\
        \begin{figure}[!ht]
            \centering
            \includegraphics[scale=0.6]{document/chapters/chapter_5/images/use_cases_satisfaction_rewards_balance.jpg}
            \caption{Contributor's Rewards Balance check}
            \label{fig:use_cases_satisfaction_rewards_balance}
        \end{figure}

        The Contributor Data retrieves, through a communication with the Billing Service, its current Rewards Balance. In order to calculate the current balance, the Billing service contacts both the Log Service (to obtain the Contribution data) and the Accounting Service (to obtain past Rewards Redemptions). The Rewards Balance is calculated by the Billing Service by summing the monetary value of the Contributions and subtracting from it the total Rewards Redemption. 

        \item \textbf{\textit{Rewards Redemption}}\\
        \begin{figure}[!ht]
            \centering
            \includegraphics[width=\linewidth]{document/chapters/chapter_5/images/use_cases_satisfaction_rewards_redemption.jpg}
            \caption{Contributor's Rewards Redemption}
            \label{fig:use_cases_satisfaction_rewards_redemption}
        \end{figure}

        The Contributor Dashboard asks the Billing Service the execution of a Rewards Redemption. First the Contributors Data Service is contacted in order to retrieve the Contributor's payment method; then, the Log Service is contacted in order to retrieve the past Contributions data. Finally, the Accounting Service is interpellated, asking for the actual operation. The Accounting Service (that can access the past Rewards Redemptions on its own), performs the calculus of the Current Balance and, if the Balance is equal or higher than Rewards Threshold and, at the same time, compatible with the requested sum to redeem, the operation is finalized contacting a Third-party Banking System.

    \end{itemize}
\end{itemize}

\subsection{Customer}
Various Customer's use cases are satisfied using processes that are very similar to the ones described in the Contributor's section; as a consequence, less dedicated diagrams will be used unless the process differs considerably from anything presented before.

\begin{itemize}
    \item \textbf{Register}\\
    The overall process is very similar to the one shown in \textit{figure \ref{fig:use_cases_satisfaction_contributor_register}}, requiring only to interchange the Contributor Dashboard with the Customer Dashboard and the Contributor Data Service with the Customer Data Service, respectively; while also the dedicated APIs invoked are necessarily different, the registration process follows the same exact logical flow.
    \vspace{10mm}
    \item \textbf{Login}
    \begin{itemize}
        \item \textbf{\textit{Login to a Dashboard}}\\
        Starting from \textit{figure \ref{fig:use_cases_satisfaction_contributor_dashboard_login}}, the use case is satisfied applying the exact same entities interchanges described in the previous use case.

        \item \textbf{\textit{Authenticate for Grid Services Invocations}}\\
        \begin{figure}[!ht]
            \centering
            \includegraphics[width=\linewidth]{document/chapters/chapter_5/images/use_cases_satisfaction_invoking_endpoint_login.jpg}
            \caption{Invoking Endpoint login}
            \label{fig:use_cases_satisfaction_invoking_endpoint_login}
        \end{figure}

        Very similar process to the one described in \textit{figure \ref{fig:use_cases_satisfaction_node_login}}; given the interchanged entities and the different APIs involved, the key difference here resides in the fact that the Customer Custom Application will implement a custom behavior that will trigger the Invoking Endpoint's login. 
    \end{itemize}
    \item \textbf{Easily integrate the Invoking Endpoint in a Customer Custom Application}\\
    \begin{itemize}
        \item \textbf{\textit{Invoke Grid Services through any Device}}\\
        As previously explained, in certain Grid Services, the Invoking Endpoint demands the coordination of the obtained Resources to another Node (e.g.: MapReduce Service); this design choice allows to extend Grid Services invocation to also low-spec devices. As far as the technological requirements are concerned, this use case requires the existence of at least one compatible Invoking Endpoint implementation for every major platform.
        The Grid Services Invocation process was previously described in \textit{figure \ref{fig:use_cases_satisfaction_node_contribution}}.

        \item \textbf{\textit{Request MapReduce service}}\\
        An Invoking Endpoint implementation will need to expose methods (through the SDK) that enable the Customer Custom Application to perform a Grid Services Invocation.

        While the specific design of the MapReduce Service will be explored in \textit{section \ref{mapreduce_service}}, \textit{figure \ref{fig:mapreduce_load_balancing}} shows a high level view of interactions in the execution of said Grid Service (continuing the example seen in \textit{figure \ref{fig:master_grid_load_balancing}}).

        \begin{figure}[!ht]
            \centering
            \includegraphics[width=\linewidth]{document/chapters/chapter_5/images/mapreduce_load_balancing.jpg}
            \caption{High level view - MapReduce Service and load balancing}
            \label{fig:mapreduce_load_balancing}
        \end{figure}

        The Invoking Endpoint connects with a Node acting as the MapReduce Master which, in turn, will act as coordinator for the execution of the MapReduce Service. Being the Contributing Nodes directly connected among each other, the Grid's load balancing (on the right) does not affect the execution.


        \begin{itemize}
            \item \textit{Define MapReduce data source}\\
            This use case requires that the SDK exposed by the Invoking Endpoint allows to define the source of the data that needs to be analyzed, meaning that the data can come from the Invoking Endpoint directly or from an external source. This is done in order to actually be able to invoke the MapReduce Service from anywhere, since not all devices are able to contain large quantities of data.

            \item \textit{Define Resource quantity to use}\\
            A Customer needs to specify, through the SDK, the quantity of Resources that it wants to reserve for the MapReduce Service execution; while having more resources certainly makes the execution faster, the Fee necessarily increases.

            \item \textit{Define Map and Reduce functions}\\
            Lastly, the Customer necessarily needs to define the Map and Reduce functions that will be utilized during the MapReduce Service execution; those are defined and provided through the usage of the Invoking Endpoint's SDK.
        \end{itemize}
    \end{itemize}
    \item \textbf{Manage requested Grid Services data through a Dashboard}\\
    \begin{itemize}
        \item \textbf{\textit{Check Grid Services Invocations history}}\\
        Very similar to \textit{figure \ref{fig:use_cases_satisfaction_past_contributions_check}}, using the Customer Dashboard and dedicated APIs instead.

        \item \textbf{\textit{Check running Grid Services Invocations}}\\
        \begin{figure}[!ht]
            \centering
            \includegraphics[width=\linewidth]{document/chapters/chapter_5/images/use_cases_satisfaction_current_grid_services.jpg}
            \caption{Running Grid Services Invocations check}
            \label{fig:use_cases_satisfaction_current_grid_services}
        \end{figure}

        The Customer Dashboard contacts the Grid Services Gateway Service in order to obtain info about the running Grid Services Invocations; then, the Grid Master is contacted which, in turn, will contact every Broker Service instance. Finally, every single Broker will broadcast the message to its connected Invoking Endpoints, asking for only the ones related to the Customer to respond. The responding Invoking Endpoints (if any), will return the data containing info about the current execution. The data will be progressively grouped and, in the end, returned as a final list to the Dashboard. 

        \vspace{10mm}
        \item \textbf{\textit{Check past Grid Services Invocations Fees}}\\
        Assimilable to the process seen in \textit{figure \ref{fig:use_cases_satisfaction_past_contributions_check}}, with the Contributor Dashboard acting as the invoking entity.

        \item \textbf{\textit{Manage payment method}}\\
        \begin{figure}[!ht]
            \centering
            \includegraphics[scale=0.85]{document/chapters/chapter_5/images/use_cases_satisfaction_contributor_payment_method.jpg}
            \caption{Contributor payment method retrieval}
            \label{fig:use_cases_satisfaction_contributor_payment_method}
        \end{figure}
        
        The management (retrieval, insertion, delete and update) of the Contributor's payment method is performed through the Contributor Dashboard collaborating with the Billing Service that, in turn, contacts the Customer Data Service, where the payment data is actually stored.
    \end{itemize}
\end{itemize}