\section{Grid computing evolution: this project's solution}
Coming from the overview on how Grid Computing works (\textit{chapter \ref{how_grid_computing_works}}), followed by the challenges and previous ideas on how to integrate mobile devices into the Grid (\textit{chapter \ref{integrating_mobile_devices_into_the_grid}}), finally concluded by an overview on the MapReduce paradigm (\textit{chapter \ref{the_mapreduce_paradigm}}), it is now time to expand this project's idea introduced in \textit{chapter \ref{introduction}}.

The solution proposed by this work takes the name of "Transparent scheduling model over heterogeneous devices"; the model aims to maintain the core characteristics of a Grid computing system while, at the same time, redesign some aspects in order to enhance and evolve its capabilities to better accommodate new necessities arising from the new technological panorama.

In particular, this Grid computing system (following the cloud computing concept) aims to offer services, ranging from simple to complex ones, to everyone that might need access to a large pool of computational resources. An example of computationally demanding distributed service is the execution of a MapReduce computation, which will be the service realized for this project's prototype.

This section, then, presents the project's aspects that differ from a traditional Grid computing approach and the reasons why it might represent a viable alternative to current Cloud computing services.

\subsection{Transparent heterogeneous devices: more contribution}
The core difference, compared to a traditional Grid approach, comes from the inclusion of mobile devices into the contributing devices base.
This project's aspect was discussed previously (\textit{section \ref{untapped_enormous_potential}} and \textit{chapter \ref{integrating_mobile_devices_into_the_grid}}) and results in a broader contribution to the Grid system coming from people that tend to primarily use their mobile devices (the vast majority nowadays).

One core characteristic emerges from the discussion of a previous work: the contributing mobile devices needs to autonomously interact with the Grid system.
This autonomy principle stands in direct contrast with the approach proposed by the work discussed in \textit{section \ref{proxy_based_cluster_architecture}}, but it is mandatory in a system where simplicity is key; moreover, the limitation of an Interlocutor device is no longer justified with the performances of current mobile devices given by the technological progress had in the recent years. It is important to specify that, while mobile devices have now better performances, technological limitations (compared to contemporary computers) still needs to be taken in account while designing the solution.

The characteristics discussed in this section influence the solution's name:
\begin{itemize}
    \item \textbf{Transparent scheduling}\\
    Traditional Grid systems operates with the assumption of the presence of a single type of contributing devices: computers; because this assumption does not hold here, a layer of abstraction needs to be put in action. A device is seen as a collection of resources that it can offer, which will be the only relevant aspect when scheduling task that said devices will need to execute to collectively complete a service offered by the Grid; hence, the devices' heterogeneity becomes transparent to the system.
    \item \textbf{Heterogeneous devices}\\
    This aspect of the solution's name is influenced not only by the fact that heterogeneous devices are supported for contribution, but also by the fact that different types of devices can also access the Grid's services (this will be discussed in \textit{section \ref{grid_services_for_all_devices}}).
\end{itemize}

\subsection{Low infrastructural costs: peer to peer philosophy}
TODO

\subsection{Toward ubiquitous computing: Grid services for all devices}\label{grid_services_for_all_devices}
TODO

\subsection{Cheaper access to cloud services: Grid services for everyone}
TODO

\subsection{Big data analysis: complex Grid services}
TODO

\subsection{Breaking the complexity barrier: broader contributing user base}
TODO

\subsection{Anticipating market trends: decreasing number of desktop devices}
TODO