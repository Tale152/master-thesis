\section{Solution analysis}
After a high level discussion about the characteristics of the solution proposed by this work, now it is time to make an additional step in the road that leads to the concretization of a working system that realizes such idea.

Here, the Domain will be explored, clarifying concepts and nomenclatures in order to uniquely identify the building blocks needed to move forward.
Then, use cases will be discussed, explaining what the system needs to do, seen from the point of view of the actors interacting with it.
Finally, the requirements will be presented, aiming to provide a clear set of criteria to satisfy for a system that wants to use the Transparent scheduling model over heterogeneous devices.

\subsection{Domain: Ubiquitous Language}
Before continuing, it is important to specify the nomenclatures and relative definitions on what composes the domain in which the projects operate; in order to do so, the Ubiquitous Language (UL) tool, coming from the Domain Driven Design discipline \cite{ddd}, will be used. From this point and forward, a term listed here will be uniquely used to identify its mapped concept:
\begin{itemize}
    \item \textbf{Contributor}\label{contributor}\\
    A person (up until this point identified as the Volunteer in Volunteer Computing) which offers one or more \textit{Contributing Endpoints} in order to perform a \textit{Contribution} finalized to the completion of \textit{Grid Services}. Through its \textit{Contribution}, it gains \textit{Rewards} that are accumulated in its \textit{Reward Balance}; once a certain \textit{Reward Threshold} is reached, a Contributor can perform a \textit{Rewards Redemption}. The Contributor can manage its data through a \textit{Contributor Dashboard}.
    \item \textbf{Contribution}\label{contribution}\\
    The automated act performed by a \textit{Contributing Endpoint} (offered by a \textit{Contributor}) in order to complete \textit{Grid Services} and, consequently, gain \textit{Rewards}.
    \item \textbf{Customer}\label{customer}\\
    An entity (whether a single person, a business, etc...) that wants to perform a \textit{Grid Service Invocation} through an \textit{Invoking Endpoint} integrated into a \textit{Customer Custom Application}. This entity can manage its data and past \textit{Service Invocations} and \textit{Fees} utilizing a \textit{Customer Dashboard}.
    \item \textbf{Grid Service Invocation}\label{grid_service_invocation}\\
    The act, requested by a \textit{Customer} through the use of an \textit{Invoking Endpoint}, of requesting a certain amount of \textit{Resources} (paying, proportionally, a \textit{Fee}) in order to perform a \textit{Grid Service}.
    \item \textbf{Grid Owner}\label{grid_owner}\\
    Entity that owns the \textit{Grid} system and is therefore responsible for its maintenance, providing all the medium to access to the \textit{Node} and \textit{Invoking Endpoint} software, as well as managing the \textit{Cloud Services}.
    \item \textbf{Grid}\label{grid}\\
    The system as a whole (owned by the \textit{Grid Owners}), based on the Grid computing principles. It offers various \textit{Grid Services} and, architecturally, it is composed by a multitude of \textit{Cloud Services}, a dynamically ever-changing number of \textit{Nodes} (provided by \textit{Contributors}) and an also variable number of \textit{Invoking Endpoints} (owned by \textit{Customers}) used to invoke its Grid Services. The parts composing the Grid are organized following the Layered Grid Architecture (\textit{section \ref{layered_grid_architecture}}).
    \item \textbf{Grid Service}\label{grid_service}\\
    A service (computation, storage, etc...) offered by the \textit{Grid} which is requested by a \textit{Customer} through an \textit{Invoking Endpoint}, paying the corresponding \textit{Fee} (which can vary depending on the amount of \textit{Resources} requested). In order to be performed, the \textit{Cloud services} coordinate the \textit{Contribution} by the \textit{Nodes} (which perform \textit{Tasks}) offered by \textit{Contributors} (who gain \textit{Rewards} for such Contributions). One particular concretization of a Grid Service is the \textit{MapReduce service}.
    \item \textbf{Task}\label{task}\\
    An abstract unit of \textit{Contribution} performed by a \textit{Node} in order to obtain the result of a \textit{Grid Service Invocation}. Depending on the particular \textit{Grid Service}, different Tasks exist.
    \item \textbf{Cloud Service}\label{cloud_service}\\
    A backend server, maintained by the \textit{Grid Owners}, operating in the Cloud. Its duties depend on its role, but the distinct traits of a Cloud Service include being owned and maintained by the \textit{Grid Owners} and the ability to coordinate the \textit{Contribution} of \textit{Nodes} participating in the \textit{Grid} and the access to \textit{Grid Services} invoked by \textit{Invoking Endpoints}. In relation to the Layered Grid Architecture (\textit{section \ref{layered_grid_architecture}}), Cloud Services all together embody the Collective Layer (\textit{section \ref{fig:collective_layer}}) as shown in \textit{figure \ref{fig:layered_grid_architecture_and_ul}}.
    \begin{figure}[!ht]
        \centering
        \includegraphics[width=\linewidth]{document/chapters/chapter_5/images/layered_grid_architecture_and_ul.png}
        \caption{UL relationship with Layered Grid Architecture}
        \label{fig:layered_grid_architecture_and_ul}
    \end{figure}
    \item \textbf{Node}\label{node}\\
    A Node is an abstraction representing a software entity providing the mechanisms used by a \textit{Contributing Endpoint} (owned by a \textit{Contributor}) to connect to the \textit{Grid} (through the \textit{Cloud Services}) in order to perform a \textit{Contribution} finalized to the completion of \textit{Grid Services} offering \textit{Resources} and executing \textit{Tasks}. Referring to the Layered Grid Architecture (\textit{section \ref{layered_grid_architecture}}), a Node can be logically mapped to the concrete realization of the Resource and Connectivity layers (\textit{section \ref{resource_layer}} and \textit{\ref{connectivity_layer}}), as well as the definition of the Fabric layer interfaces (\textit{section \ref{fig:fabric_layer}}) which will be concretely implemented in the \textit{Contributing Endpoint} (\textit{figure \ref{fig:layered_grid_architecture_and_ul}}).
    \item \textbf{Contributing Endpoint}\label{contributing_device}\\
    An abstraction representing a client software running on a device (Computer, Smartphone or Tablet), owned by a \textit{Contributor}, which is able to perform a \textit{Contribution} finalized to completing \textit{Grid Services}; such \textit{Contribution} is performed by the software (running on the Contributing Endpoint) that is built interfacing with the \textit{Node} software abstraction. Not all Contributing Endpoints are able to perform every \textit{Grid Service} that can exist; such ability is determined by the \textit{Resources} that the specific Contributing Endpoint can offer. A Contributing Endpoint can be logically mapped (\textit{figure \ref{fig:layered_grid_architecture_and_ul}}) to the concrete implementation of the Fabric Layer (\textit{section \ref{fig:fabric_layer}}) of the Layered Grid Architecture (\textit{section \ref{layered_grid_architecture}}).
    \item \textbf{Resource}\label{resource}\\
    An abstraction representing what a \textit{Contributing Endpoint} can offer (through the \textit{Node} software) in order to offer a \textit{Contribution} to the completion of a \textit{Grid Service}. A resource can belong to one of the following types:
    \begin{itemize}
        \item Hardware (processor, memory, disk, etc...)
        \item Software (OS, supported programming languages, compatible \textit{Tasks}, etc...)
        \item Network (connection type, download speed, upload speed, etc...)
    \end{itemize}
    \item \textbf{Invoking Endpoint}\label{invoking_endpoint}\\
    An abstraction representing a software entity providing the mechanisms used by an \textit{Invoking Device} (owned by a \textit{Customer}) in order to perform a \textit{Grid Service Invocation}. This abstraction is used in a \textit{Customer Custom Application}, integrating the \textit{Grid Services} into such application. Continuing the mapping (\textit{figure \ref{fig:layered_grid_architecture_and_ul}}) to the fundamental Layered Grid Architecture (\textit{section \ref{layered_grid_architecture}}), an Invoking Endpoint (together with the \textit{Customer Custom Application}), embodies the Application Layer (\textit{section \ref{fig:application_layer}}).
    \item \textbf{Customer Custom Application}\label{customer_custom_application}\\
    Software application, developed by the \textit{Customer}, that uses the \textit{Invoking Endpoint} software abstraction in order to integrate \textit{Grid Services} into such custom application ran on an \textit{Invoking Device}.
    \item \textbf{Invoking Device}\label{invoking_device}\\
    A device that runs the \textit{Customer Custom Application}. Such device can be owned by the \textit{Customer} itself or by people using the \textit{Customer Custom Application} that the \textit{Customer} is commercializing. In regard to the \textit{Grid}, the focal point here is that it is a device that is performing a \textit{Grid Service Invocation} through the \textit{Customer Custom Application} that is utilizing the \textit{Invoking Endpoint} software.
    \item \textbf{Contributor Dashboard}\label{contributor_dashboard}\\
    A software entity through which a \textit{Contributor} can manage its \textit{Contributing Endpoints} and check its \textit{Rewards Balance} (eventually performing a \textit{Rewards Redemption}).
    \item \textbf{Customer Dashboard}\label{customer_dashboard}\\
    A software entity through which a \textit{Customer} can manage its present and past \textit{Grid Service Invocations} and check the relative \textit{Fees} as well as registering a payment method used to pay such \textit{Fees}.
    \item \textbf{MapReduce Service}\label{mapreduce_service}\\
    A particular incarnation of a \textit{Grid Service} that performs a MapReduce computation over the \textit{Grid} utilizing a \textit{Node} as \textit{MapReduce Master} and several nodes as \textit{Map Workers} and \textit{Reduce Workers}.
    \item \textbf{MapReduce Master}\label{mapreduce_master}\\
    Role taken by a \textit{Node} in the execution of a \textit{MapReduce Service} its roles follow the ones described in \textit{section \ref{execution_flow}}.
    \item \textbf{Map Worker}\label{map_worker}\\
    A \textit{Node}, coordinated by the \textit{MapReduce Master}, that executes \textit{Map Tasks} and send the results to the correct \textit{Reduce Worker}.
    \item \textbf{Map Task}\label{map_task}\\
    A concrete \textit{Task} performed by a \textit{Map Worker}. A Map function provided by the \textit{Customer} (obtained through the \textit{MapReduce Master} coordination) is executed on data of which the location is specified also by the \textit{Consumer}.
    \item \textbf{Reduce Worker}\label{reduce_worker}\\
    A \textit{Node}, coordinated by the \textit{MapReduce Master}, that executes \textit{Reduce Tasks} on data received by \textit{Map Workers}.
    \item \textbf{Reduce Task}\label{reduce_task}\\
    A concrete \textit{Task} performed by a \textit{Reduce Worker}. A Reduce function provided by the \textit{Customer} (obtained through the \textit{MapReduce Master} coordination) is executed on data received from \textit{Map Workers}.
    \item \textbf{Fee}\label{fee}\\
    A monetary amount that the \textit{Customer} needs to pay in order to perform a \textit{Grid Service Invocation}. The amount is influenced by the type of service requested and the quantity of resources employed to complete such service.
    \item \textbf{Reward}\label{reward}\\
    A monetary compensation for the \textit{Contributor}, calculated based on the \textit{Fee} paid by the \textit{Customer} (minus the \textit{Grid Owner} profit) proportionally to the \textit{Contribution} performed by a \textit{Contributing Device} in a \textit{Grid Service Invocation}. Every new Reward contributes to the \textit{Reward Balance}.
    \item \textbf{Rewards Balance}\label{rewards_balance}\\
    The sum of the \textit{Rewards} obtained by the \textit{Contributor} for the \textit{Contribution} of the \textit{Contributing Endpoints} owned by it. This balance can be lowered when the \textit{Contributor} performs a \textit{Rewards Redemption}. 
    \item \textbf{Rewards Redemption}\label{rewards_redemption}\\
    An action performable by the \textit{Contributor} (only if the \textit{Rewards Balance} is currently higher or equal to the \textit{Rewards Threshold}) that lowers the value of the \textit{Rewards Balance} allowing the \textit{Contributor} to transfer that monetary value by using a payment method.
    \item \textbf{Rewards Threshold}\label{rewards_threshold}\\
    The value that needs to be reached in order to allow the \textit{Contributor} the possibility to perform a \textit{Rewards Redemption}.
    \item \textbf{Access policies}\label{access_policies}\\
    \textit{Contribution Endpoint}-specific Access policies defined by the \textit{Contributor}, specifying which \textit{Resources} are offered to the \textit{Grid} for \textit{Contribution}.
\end{itemize}

\subsection{Use cases}
TODO

\begin{figure}[!ht]
    \centering
    \includegraphics[width=\linewidth]{document/chapters/chapter_5/images/contributor_use_cases.jpg}
    \caption{Contributor's use cases diagram}
    \label{fig:use_cases_contributor}
\end{figure}

\begin{figure}[!ht]
    \centering
    \includegraphics[width=\linewidth]{document/chapters/chapter_5/images/customer_use_cases.jpg}
    \caption{Customer's use cases diagram}
    \label{fig:use_cases_customer}
\end{figure}

\subsection{Requirements}
TODO Moscow explanation and requisite type mapping (and types explanation).

\subsubsection{Must have}
All the requirements that are necessary for the successful completion of a complete production release.

\paragraph{Contributor side}
\begin{itemize}
    \item \textbf{Registration and identification}
    \begin{itemize}
        \item The Contributor must be able to register to the Grid system in order to uniquely identify contributions regarding its devices participating in the Grid.
        \item Once authenticated to a Contributing Endpoint, the Contributor must not be requested to authenticate again unless it is absolutely necessary.
    \end{itemize}
    \item \textbf{Contribution}
    \begin{itemize}
        \item It must be possible to contribute to the Grid using any device among Computers, Smartphones and Tablets running the most popular respective Operative Systems.
        \item Configuration for the Contributor must be easy, guided and fast (5 minutes maximum).
        \item Every device that the Contributor offers must be configured just once and require no additional efforts unless the Contributor chooses to change configurations.
    \end{itemize}
    \item \textbf{Rewards}
    \begin{itemize}
        \item Every successful contribution to the execution of a Grid service must be registered contributing to the rewards balance of the Contributor.
        \item The Contributor must be able to see its current rewards balance.
        \item The Contributor must be able to see past contributions in which its devices contributed to the execution of a certain Grid service.
        \item Once a certain monetary threshold is reached, the Customer must be able to withdraw its rewards from its balance.
    \end{itemize}
    \item \textbf{Security}
    \begin{itemize}
        \item Execution environment-specific security measures must be implemented to protect the Contributor.
    \end{itemize}
    \item \textbf{User experience}
    \begin{itemize}
        \item The application running on mobile devices (used to contribute to the Grid) must be aware of battery status in order not to drain battery in low-battery situations.
        \item The application running on mobile devices (used to contribute to the Grid) must be aware of the type of connection used in order not to drain data from the Consumer mobile plan (unless the Consumer allows this possibility).
    \end{itemize}
\end{itemize}
\paragraph{Customer side}
\begin{itemize}
    \item \textbf{Registration and identification}
    \begin{itemize}
        \item The Customer must be able to register to access Grid's services.
        \item The Customer must go through a verification process in order to be reliable and accountable before being able to use Grid's services.
    \end{itemize}
    \item \textbf{Payments}
    \begin{itemize}
        \item The Customer must be able to register its payment method once and use it every time it invokes one Grid's service.
        \item The Customer must pay individual Grid's services invocations.
    \end{itemize}
    \item \textbf{Framework utilization}
    \begin{itemize}
        \item The Customer must be able to integrate the Framework in its own project to utilize the Grid's services.
        \item The Customer must be able to invoke Grid’s services from any adequate device (Computer, Tablet or Smartphone) using the integrated Framework.
        \item The Customer must be able to easily increase/decrease the amount of Grid's resources tapped into, removing the underlying complexity.
    \end{itemize}
    \item \textbf{MapReduce service utilization}
    \begin{itemize}
        \item The Customer must be able to implement its own Map and Reduce functions in a MapReduce service invocation.
        \item The MapReduce parameters such as number of Map and Reduce workers must be configurable by the Customer.
        \item The Customer must be able to define the data source for the MapReduce execution (whether from the invoking device or an external source).
    \end{itemize}
\end{itemize}
\paragraph{Project side}
\begin{itemize}
    \item \textbf{Scalability}
    \begin{itemize}
        \item The Grid system must be able to easily scale horizontally in a fully automated way.
        \item Grid backend services must be as loosely coupled as possible, resulting in a modular composition of technologically independent services that operate exchanging messages over defined channels.
        \item It must be possible to create on the go (depending on needs dictated by the current traffic) multiple instances of the Grid backend services, employing a mechanism to discover and contact such new instances.
    \end{itemize}
    \item \textbf{Efficiency}
    \begin{itemize}
        \item Employed resources for running the Grid system must be adequate to the load that the system has to currently handle, dynamically adjusting to cut expenses.
        \item The Grid system must be geographically-aware and manage node utilization taking in account physical distance in order to reduce latency, improving performances.
        \item When possible, services must be implemented in a way that reduces traffic to the Grid’s systems, moving such traffic to the contributing Nodes.
        \item Every execution of a service offered by the Grid must use Nodes that have sufficient capabilities to complete the task adequately.
    \end{itemize}
    \item \textbf{Compatibility}
    \begin{itemize}
        \item The Framework developed for Customers and the Core of the Contribution application for Contributors must be implemented using a technology that can be easily integrated in all major Operative Systems that run on the target devices.
    \end{itemize}
    \item \textbf{MapReduce}
    \begin{itemize}
        \item The MapReduce execution must grant an adequate level of reliability and fault tolerance.
        \item The MapReduce tasks on the Node side must be designed in a way that allows to integrate support for new programming  languages for the execution of Customer-defined Map and Reduce functions.
    \end{itemize}
    \item \textbf{Expandability and maintainability}
    \begin{itemize}
        \item The solution must be designed in a way that allows to easily expand the capabilities of the Grid, allowing the creation of newer services that can be executed using the Nodes.
        \item It must be possible to parallelize implementation of features and maintenance, structuring the solution using a microservices approach allowing multiple teams to work on multiple sub-portions of the project.
    \end{itemize}
    \item \textbf{Security}
    \begin{itemize}
        \item Communications between the components of the Grid must be cyphered and use other security measures.
        \item There must be a permission system used to manage access to a Node resource.
        \item Operations inside the Grid must be logged in order to have a fully comprehensive view of every movement that happens inside the system.
    \end{itemize}
\end{itemize}

\subsubsection{Should have}
Requirements that are important but not necessary for a production release and can be postponed to subsequent releases.

\paragraph{Contributor side}
\begin{itemize}
    \item It should be possible to see the status of every device owned by the Contributor.
    \item It should be possible, from the Contributing Endpoint, to see the local current status.
    \item It should be possible to limit the access to certain resources of the device.
    \item It should be possible to manually enable/disable the contribution of a device.
    \item It should be possible to schedule a trigger (time and/or events) to start/stop the contribution.
    \item The applications used by the Contributors should limit usage of resources if the user is actively using the device in order to not have a negative impact on user experience (avoiding incentives for the user to uninstall the application).
\end{itemize}
\paragraph{Customer side}
\begin{itemize}
    \item There should be a tool for testing the Map and Reduce functions.
    \item It should be possible to specify the minimum capabilities of the devices involved in the execution of a Grid service.
\end{itemize}
\paragraph{Project side}
\begin{itemize}
    \item There should be device-specific optimizations.
\end{itemize}

\subsubsection{Could have}
Requirements that are nice to have but have a much smaller impact when left out of the release.
\paragraph{Customer side}
\begin{itemize}
    \item There could be a standardized MapReduce web tool that enables less specialized organizations (from a technological point of view) to perform operations constructed by a facilitated GUI.
\end{itemize}

\subsubsection{Won't have}
Requirements that have been recognized as not a priority for the release timeframe.
\paragraph{Contributor side}
\begin{itemize}
    \item Development of a browser-based contribution limited to certain types of Grid services.
\end{itemize}
\paragraph{Project side}
\begin{itemize}
    \item Development of additional Grid services other than MapReduce
\end{itemize}