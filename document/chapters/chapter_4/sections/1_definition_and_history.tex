\section{Definition and history}
\textbf{MapReduce} is a \textbf{popular programming model} designed to \textbf{easily process and generate large datasets on clusters of commodity machines}. Through this paradigm, a computation can be expressed in terms of a \textbf{map and reduce functions} while \textbf{the underlying system deals with communication, parallelization and error handling}, making it easy to use even for programmers who have no experience with distributed systems.

\textbf{Google} created this paradigm in \textbf{2003} in order to reduce development time and cost on their projects; after an analysis of the problem, Google's engineers noticed how \textbf{the majority of the computations in their products could be expressed through the map and reduce abstractions} that are typically present in \textbf{functional languages}. The MapReduce paradigm has been \textbf{used in a variety of Google's project}, including the indexing system used by the Google search engine \cite{google_mapreduce}.

\textbf{\href{https://hadoop.apache.org/}{Apache Hadoop}}, inspired by Google's work, integrates the MapReduce paradigm in its free-licensed framework, making it \textbf{one of the most used options} when it comes to applying distributed computations following this paradigm.
