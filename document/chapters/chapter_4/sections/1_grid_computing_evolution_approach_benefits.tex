\section{Grid computing evolution: this project's solution}
Coming from the overview on how Grid Computing works (\textit{chapter \ref{how_grid_computing_works}}), followed by the challenges and previous ideas on how to integrate mobile devices into the Grid (\textit{chapter \ref{integrating_mobile_devices_into_the_grid}}), it is now time to expand this project's idea, already introduced in \textit{chapter \ref{introduction}}.

The solution proposed by this work takes the name of \textbf{\textit{"Transparent scheduling model over heterogeneous devices"}}; the model aims to \textbf{maintain} the \textbf{core characteristics of a Grid computing system} while, at the same time, \textbf{redesign some aspects} in order to enhance and evolve its capabilities \textbf{to better accommodate new necessities arising from the current technological panorama}.

In particular, this Grid computing system (following the Cloud Computing modern approach) aims to \textbf{offer services}, ranging from simple to complex ones, \textbf{to everyone that might need access to a large pool of computational resources}.

\textbf{An example} of computationally demanding distributed service is the execution of a \textbf{MapReduce computation}, which will be the service \textbf{realized for this project's prototype}.

\begin{info}
    See \textit{appendix \ref{the_mapreduce_paradigm}} for an overview on the MapReduce paradigm.
\end{info}

This section, then, presents the \textbf{project's aspects that differ from a traditional Grid computing approach} and the reasons \textbf{why it might represent a viable alternative to current Cloud computing services}.

\subsection{Transparent heterogeneous devices: more contribution}\label{transparent_heterogeneous_devices_more_contribution}
The \textbf{core difference}, compared to a traditional Grid approach, comes from the \textbf{inclusion of mobile devices into the contributing devices base}.
This project's aspect was discussed previously (\textit{section \ref{untapped_enormous_potential}} and \textit{chapter \ref{integrating_mobile_devices_into_the_grid}}) and \textbf{results in a broader contribution to the Grid system coming from people that tend to primarily use their mobile devices} (the vast majority nowadays).

One core characteristic emerges from the discussion of a previous work: \textbf{the contributing mobile devices need to autonomously interact with the Grid system}.
This autonomy principle stands in direct contrast with the approach proposed by the work discussed in \textit{section \ref{proxy_based_cluster_architecture}}, but it is mandatory in a system where simplicity is key; moreover, \textbf{the limitation of an Interlocutor device is no longer justified with the performances of current mobile devices} given by the technological progress of the recent years. It is important to specify that, while mobile devices have now better performances, \textbf{technological limitations} (compared to contemporary computers) \textbf{still need to be taken into account while designing the solution}.
The characteristics discussed in this section influence the solution's name:
\begin{itemize}
    \item \textbf{Transparent scheduling}\\
    \textbf{Traditional Grid systems operate with the assumption of the presence of a single type of contributing devices}: computers; \textbf{since this assumption does not hold here, a layer of abstraction needs to be put into action}. \textbf{A device is seen as a collection of resources that it can offer}, which will be the only relevant aspect when scheduling tasks that said devices will need to execute to collectively complete a service offered by the Grid; hence, \textbf{the devices' heterogeneity becomes effectively transparent to the system}.
    \item \textbf{Heterogeneous devices}\\
    This aspect of the solution's name is \textbf{influenced not only by the fact that heterogeneous devices are supported for contribution, but also by the fact that different types of devices can also access the Grid's services} (this will be discussed in \textit{section \ref{grid_services_for_all_devices}}).
\end{itemize}

\subsection{Low infrastructural costs: Volunteer computing philosophy}\label{low_infrastructural_costs_volunteer_computing_philosophy}
When it comes to \textbf{Cloud computing}, \textbf{one critical problem is the amount of resources} (hardware and, consequently, economical) \textbf{needed to maintain up and running the infrastructure required} to perform the services that the Cloud computing platform wants to offer; \textbf{Grid computing combined with Volunteer computing} (\textit{section \ref{history_and_applications_using_the_grid}}) \textbf{vastly reduces this problem}.

A typical Cloud computing platform operates, in simple terms, by a multitude of hosted machines (running the required software) communicating among each other; the maintenance costs of said machines is completely sustained by the Cloud computing platform owner. \textbf{With Volunteer computing}, on the other hand, \textbf{the machines that compose the Grid architecture are divided in two categories}:
\begin{itemize}
    \item \textbf{Maintenance costs sustained by the Grid owner}\\
    Here reside the machines running the core coordination mechanisms of the Grid system. The instances of said machines can be dynamically increased or decreased depending on the necessities dictated by the current workload, resulting in paying the strict necessary to make the system work at any given time.
    \item \textbf{Maintenance costs sustained by the Volunteers}\\
    This category comprehends the actual machines (Computers, Smartphones and Tablets) that cooperate in order to execute the services provided by the Grid. Since the devices are owned and used by the Volunteers, the vast majority of the costs associated with said devices are demanded to them. Said costs are represented mainly by the device maintenance, the electricity needed to make them work, and the Internet traffic generated; but, as discussed in \textit{sections \ref{mobile_users_common_behavior} and \ref{economically_sustainable_internet_connections}}, these costs will be sustained by the Volunteer anyway (especially in the case of mobile devices), regardless contributing to the Grid or not. Furthermore, the Volunteer receives a compensation for its contribution (\textit{section \ref{fair_share_business_model}}), making it actually convenient for them to contribute.
\end{itemize}

On a final note, this approach also has a positive effect on the environment: the vastly diffused Cloud model results in a great consumption of electricity to run the dedicated machines composing the infrastructure; on the other hand, \textbf{demanding the execution of software into Volunteer machines} (that would be turned on anyway) \textbf{reduces the environmental impact derived by the electricity consumption}.

\subsection{Removing the complexity: broader contributing user base}
Although \textbf{Volunteer computing} has been a concept for many years, \textbf{the volunteering was typically done by people with relative confidence in using technology}, in particular (necessarily) computers.

It is not uncommon that the typical computer user, for example, does not have the knowledge (or the confidence) to find and launch a Windows installer or, even worse, install a program from an integrated manager on any Linux distribution. But, as technology increased in diffusion, \textbf{in recent years a broader audience has become familiar with the concept of installing Apps through the use of a store} (mainly the App Store and the Play Store), \textbf{making it easier to reach new users}. Using such established means of distribution makes it \textbf{realistic to also reach potential Volunteers that mainly use mobile devices} and may not have the technological skills to contribute using a computer. Furthermore, the "store" approach on mobile devices has recently generated similar stores on the desktop OSes thus creating a \textbf{new simplified distribution medium} (even if still not vastly used today).

But, even if the first barrier on reaching users is removed, it is important to keep in mind the \textbf{simplicity principle} behind the idea, ensuring that the Volunteer has to perform a \textbf{setup as easy and guided as possible} in order to \textbf{configure its device once and then never have to do that again}.

\subsection{Toward ubiquitous computing: Grid services for all devices}\label{grid_services_for_all_devices}
The "\textbf{\textit{heterogeneous}}" \textbf{keyword} in the solution's name \textbf{refers not only to the possibility of using a multitude of different devices in order to contribute to the Grid, but it also includes the feature of accessing Grid's services from computers as well as smartphones and tablets}.

The previous work described in \textit{section \ref{mobile_to_grid_middleware}} introduced this idea with the use of a middleware machine; a subsequent work (\textit{section \ref{autonomous_mobile_middleware}}) improved this idea by removing the necessity of such middleware. This approach mirrors the one previously described for devices' contribution to the Grid and better suits this solution, obtaining broader devices' compatibility from both ends of the execution of a service (requestor machine side and executor machines side). \textbf{As a primary consequence, a multitude of applications running on various devices can build complex behaviors with the support of the Grid's resources}.

The possibility of executing computationally heavy tasks even from low-performance devices is a \textbf{step toward the ubiquitous computing idea} for the future, having \textbf{great versatility and computing power accessed by day-to-day objects}.

\subsection{Cheaper access to Cloud services: Grid services for everyone}\label{cheaper_access_to_cloud_services}
\textbf{As a result of having broad compatibility for devices' contribution} (\textit{section \ref{transparent_heterogeneous_devices_more_contribution}}) \textbf{and low infrastructural costs} (\textit{section \ref{low_infrastructural_costs_volunteer_computing_philosophy}}), \textbf{the cost of accessing to the Grid services} (that can be seen as Cloud computing services from the customer point of view) \textbf{becomes cheaper} than utilizing other platforms that run their infrastructures hosting their private machines.

\textbf{Realistically, performances of services ran on Grid computing systems will be slightly worse compared to services that host their dedicated machines}; this is mainly because of the potential faults arising from the less stable connection of the devices. \textbf{Nonetheless, this is a highly viable and much cheaper alternative in domains where performances are not a highly strict requirement} and a little delay can be accepted, opening the possibility of using such services also to entities that do not have a lot of financial resources.

\vspace{10mm}
Expanding the considerations seen up until now, \textbf{the Grid owners get value and a revenue from the Customers that}, at the same time, \textbf{save money accessing cheaper services}. \textbf{On the Volunteer side, there is a compensation} for doing absolutely nothing more than easily configuring a device once \textbf{and the environment is less impacted} while still performing advanced computations. In conclusion, \textbf{everyone wins}. 

\subsection{Anticipating market trends: current devices' panorama}
As discussed in \textit{section \ref{from_pdas_to_smartphones}}, \textbf{the current market trend shows how computer sales are decreasing} and, as experts say \cite{smartphones_sales}, this trends is not expected to reverse. In light of these considerations, \textbf{as the number of computers decreases while other devices increase} (mainly mobile), \textbf{it becomes increasingly important to think computational solutions around devices that will actually be used}.

The solution proposed, although already relevant, progressively increases its significance with time with the gradual but inevitable global shift of device types composition.