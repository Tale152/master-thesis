\chapter{The MapReduce paradigm}
As previously discussed, Grid computing is a multipurpose tool aimed to offer devices' resources to perform various types of tasks. This project, despite being multipurpose in nature itself (meaning that organizes its structure keeping in mind the possibility to support additional services in the future), aims to specifically offer to a user the execution of a MapReduce service over the Grid.

Hence, this short chapter discusses the MapReduce paradigm, concluding the overview of the problem before exploring the solution proposed by this work. Firstly, a definition and the history of such paradigm will be provided. Then, the programming model will be discussed, explaining the basic concepts needed to utilize MapReduce; following that, an overview of the master-slave architecture is provided. Continuing the chapter, in an attempt to clarify as much as possible how the actual execution of a program that employs MapReduce works, an easy example of a MapReduce computation is displayed. To conclude this chapter, an analysis of the execution flow is presented.

\input{document/chapters/chapter_4/sections/1_definition_and_history.tex}
\section{The programming model}
In order to execute a MapReduce computation, it is required, as \textbf{input}, \textbf{a set of key/value pairs}. Said values are \textbf{modified through the Map and Reduce functions} and, ultimately, produce as \textbf{output another set of key/value pairs}. 

The Map and Reduce functions are \textbf{written by the user} but in the background, through the framework, behave in the following way:
\begin{itemize}
    \item \textbf{Map}: \textit{(k1, v1) $\Longrightarrow$ list(k2, v2)}\\
    The map function takes a single pair as input and produces a set of intermediate key/value pairs. The framework automatically merges the intermediate sets, grouping them using the keys. Said values are then passed as input to the Reduce function.
    \item \textbf{Reduce}: \textit{(k2, list(v2)) $\Longrightarrow$ list(v2)}\\
    The Reduce function uses the input provided by the automatic merge performed by the framework; every Reduce execution takes a pair of composed of the intermediate key and a collection of values associated to that key. Said pairs are provided using an iterator in order to work with collections that are too large to fit in memory. The values associated to the key are merged to form a possibly smaller set of values, resulting typically in one or zero output values produced as result (even though the function produces a list of values).
\end{itemize}
A programmer that implements a computation following this paradigm does not need to provide anything else but, \textbf{behind the scenes}, the framework performs additional operations such as \textbf{Splitting} (that divides the input in smaller parts to be executed on the multiple workers), \textbf{Shuffling} (that merges the output of the individual Map functions) and \textbf{Collect} (that reunites results produced by the various workers).

Even though the functioning of this paradigm is conceptually easy, the next section displays a concrete example in order to clarify the topic discussed by this section.
\section{Architecture}
TODO

\begin{figure}[H]
    \centering
    \includegraphics[scale=0.45]{document/chapters/chapter_4/images/hadoop_master_slave_architecture.png}
    \caption{Hadoop's MapReduce master-slave architecture\cite{hadoop_map_reduce}}
    \label{fig:hadoop_master_slave_architecture}
\end{figure}
\section{Example}
In this example (\textit{figure \ref{fig:mapreduce_example}}), a simple word count will be performed: given a source of text, the algorithm will count the occurrences of every word appearing in such text.

\begin{figure}[H]
    \centering
    \includegraphics[width=\linewidth]{document/chapters/chapter_4/images/mapreduce_example.png}
    \caption{Example of word count expressed through the MapReduce paradigm \cite{mapreduce_example_site}}
    \label{fig:mapreduce_example}
\end{figure}

The process begins with splitting the input in smaller portions, in this case the text is split by row. On the resulting portions, the Map function is performed; in this example, every word (that is used as a key) is mapped to the value "1" (since it is one appearance of said word). The results are then grouped through the Shuffling operation, using the key as the grouping criteria. Finally, the Reduce function is executed, in different machines, on every group (the example sums the values in order to get the final number of occurrences for every word); the results of the Reduce operations is then collected, obtaining the final collection of key/value pairs.
\section{Execution flow}
TODO

\begin{figure}[H]
    \centering
    \includegraphics[width=\linewidth]{document/chapters/chapter_4/images/mapreduce_execution_flow.png}
    \caption{MapReduce execution flow \cite{google_mapreduce}}
    \label{fig:mapreduce_execution_flow}
\end{figure}