\section{Goals and context}
After having engineered a complete system that satisfies the use cases defined in \textit{section \ref{use_cases}}, it is now time to create a prototype that brings a portion of it to reality. The prototype takes the name of the "Interconnected project".

Before explaining the work behind this prototype, a few premises have to be clarified in order to understand some choices taken during the development.

First off, the focus of this work is to discuss the topic of integrating mobile devices in a Grid system and, most importantly, engineering a solution that defines a blueprint for actually bring the idea to reality. Due to a lack of resources (economical, time, equipment, team members, etc...), the prototype does not try to realize the whole system defined in the previous chapter, but only a subset of core functionalities with the goal to demonstrate that the core idea is also technologically feasible.

As a direct consequence of that, the prototype will not realize aspects that, while certainly important in a real product, do possess already well established technologies (authentication, data storage, server instances replication, payment systems, etc...), but only the innovative aspects of the project.
