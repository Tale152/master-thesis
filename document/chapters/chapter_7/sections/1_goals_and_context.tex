\section{Goals and context}
After having engineered a complete system that satisfies the use cases defined in \textit{section \ref{use_cases}}, it is now time to create a \textbf{prototype} that brings a portion of it to reality. The prototype takes the name of the \textbf{"\textit{Interconnected project}"}.

Before explaining the work behind this prototype, \textbf{a few premises} have to be clarified in order to understand some choices taken during the development.

Firstly, the \textbf{focus of this work} is to \textbf{discuss the topic of integrating mobile devices in a Grid system} and, most importantly, \textbf{engineering a solution that defines a blueprint for actually bringing the idea to reality}. \textbf{Due to a lack of resources} (economical, time, equipment, team members, etc...), \textbf{the prototype does not try to realize the whole system} defined in the previous chapter, \textbf{but only a subset of core functionalities} with the goal to \textbf{demonstrate that the core idea is also technologically feasible}.

As a direct consequence of that, \textbf{the prototype will not realize aspects that}, while certainly important in a real product, \textbf{do possess already well-established technologies} (authentication, data storage, server instances replication, payment systems, etc...), \textbf{focusing only on the innovative aspects} of the project.
