\section{Intrinsic challenges of Grid Computing}
While designing a Grid system, intrinsic challenges have to be taken in consideration; that means that these challenges exist in Grid Computing independently of the inclusion of mobile devices or operating with just desktop computers. This section presents a list of the main challenges that a Grid system must face.
\vspace{20mm}

\subsection{Interoperability}
As mentioned in \textit{section \ref{layered_grid_architecture}}, interoperability among any potential participant is the central issue when designing a Grid Architecture, requiring a meticulous focus on designing reliable and versatile protocols for interaction.

Unless the system is designed with the limitation of utilizing a set of identical machines, which is not desirable, machines connected to a Grid operate with \textbf{different hardware}; this differentiation requires a layer of abstraction that separates the functioning of the Grid from the interaction with a single physical node that will use a specific implementation of the standardized Fabric layer.

But machines not only offer different hardware, they differentiate each other in general by the resource they offer. In the context of distributed execution of computational tasks, for example, nodes offer \textbf{different programming languages} that they are capable of executing. This becomes a problem for interoperability since a requestor might want to execute some task with an implementation in some specific programming language to respect a performance requirement or just because that language offers useful libraries for that particular task. While an implementation of a Fabric layer using a language that is capable of running on almost any device (such as JavaScript) is certainly possible, this creates a great limitation in the Grid's use cases related to distributed computing while at the same time constraining the architecture to a specific technology, which is bad while designing any software.

Obviously, an important interoperability issue comes from the fact that \textbf{nodes can utilize any operating system}, implying different ways of installing the software necessary to run the node's tasks and, most importantly, a differentiation with the interaction with the file system and the security mechanisms specific for that particular OS.

Another resource where nodes differ is the network capabilities that they can offer. Even though this is not a differentiation, in a narrow sense, that comes from the machine in itself but from the environment where it operates, this still becomes an interoperability problem inside a Grid system; for this reason the architecture has to be designed taking in consideration that different nodes will offer \textbf{different levels of reliability and speed for their network connection}, requiring mechanisms for handling errors, disconnections and distributing work accordingly to the network's speed of the node. 

\subsection{Security}
TODO

\subsection{Geographical distribution}
TODO

\subsection{Scalability and load balancing}
TODO

\subsection{Fault tolerance and Quality of Service (QoS)}
TODO

\subsection{Dynamic sharing relationships}
TODO

\subsection{Resources discovery and selection}
TODO

\subsection{Job scheduling, replication, migration and monitoring}
TODO
